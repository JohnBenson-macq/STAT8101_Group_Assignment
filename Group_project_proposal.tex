% Options for packages loaded elsewhere
\PassOptionsToPackage{unicode}{hyperref}
\PassOptionsToPackage{hyphens}{url}
%
\documentclass[
  12pt,
]{article}
\usepackage{amsmath,amssymb}
\usepackage{iftex}
\ifPDFTeX
  \usepackage[T1]{fontenc}
  \usepackage[utf8]{inputenc}
  \usepackage{textcomp} % provide euro and other symbols
\else % if luatex or xetex
  \usepackage{unicode-math} % this also loads fontspec
  \defaultfontfeatures{Scale=MatchLowercase}
  \defaultfontfeatures[\rmfamily]{Ligatures=TeX,Scale=1}
\fi
\usepackage{lmodern}
\ifPDFTeX\else
  % xetex/luatex font selection
  \setmainfont[]{Times New Roman}
\fi
% Use upquote if available, for straight quotes in verbatim environments
\IfFileExists{upquote.sty}{\usepackage{upquote}}{}
\IfFileExists{microtype.sty}{% use microtype if available
  \usepackage[]{microtype}
  \UseMicrotypeSet[protrusion]{basicmath} % disable protrusion for tt fonts
}{}
\makeatletter
\@ifundefined{KOMAClassName}{% if non-KOMA class
  \IfFileExists{parskip.sty}{%
    \usepackage{parskip}
  }{% else
    \setlength{\parindent}{0pt}
    \setlength{\parskip}{6pt plus 2pt minus 1pt}}
}{% if KOMA class
  \KOMAoptions{parskip=half}}
\makeatother
\usepackage{xcolor}
\usepackage[left=2.54cm,right=2.54cm,top=2.54cm,bottom=2.54cm]{geometry}
\usepackage{graphicx}
\makeatletter
\def\maxwidth{\ifdim\Gin@nat@width>\linewidth\linewidth\else\Gin@nat@width\fi}
\def\maxheight{\ifdim\Gin@nat@height>\textheight\textheight\else\Gin@nat@height\fi}
\makeatother
% Scale images if necessary, so that they will not overflow the page
% margins by default, and it is still possible to overwrite the defaults
% using explicit options in \includegraphics[width, height, ...]{}
\setkeys{Gin}{width=\maxwidth,height=\maxheight,keepaspectratio}
% Set default figure placement to htbp
\makeatletter
\def\fps@figure{htbp}
\makeatother
\setlength{\emergencystretch}{3em} % prevent overfull lines
\providecommand{\tightlist}{%
  \setlength{\itemsep}{0pt}\setlength{\parskip}{0pt}}
\setcounter{secnumdepth}{5}
\ifLuaTeX
  \usepackage{selnolig}  % disable illegal ligatures
\fi
\IfFileExists{bookmark.sty}{\usepackage{bookmark}}{\usepackage{hyperref}}
\IfFileExists{xurl.sty}{\usepackage{xurl}}{} % add URL line breaks if available
\urlstyle{same}
\hypersetup{
  hidelinks,
  pdfcreator={LaTeX via pandoc}}

\author{}
\date{\vspace{-2.5em}}

\begin{document}

\textbf{Exploring the Determinants of Developer Compensation: Insights
from the Stack Overflow Annual Developer Survey 2023}

\textbf{Names of the Students in the Group:}

\begin{itemize}
\item
  Ayan Krishna Paul {[}47697296{]}
\item
  David Kong {[}47458992{]}
\item
  John Michael Benson {[}47906278{]}
\item
  Katherine Koslow {[}45649154{]}
\item
  Nergis Ilhan {[}47493895{]}
\item
  Rohan Junaid Khan {[}47843276{]}
\end{itemize}

\textbf{The Survey Data Set Selected:} The Stack Overflow Annual
Developer Survey 2023 is a comprehensive survey that gathers valuable
insights from a large and diverse population of software developers
worldwide. The survey covers a wide array of topics, including developer
roles, programming languages, tools, frameworks, job satisfaction,
career aspirations, demographic characteristics and salaries. With a
focus on capturing the evolving developer experience and understanding
the impact of emerging technologies like AI/ML on developers' workflows,
the survey offers a unique opportunity to investigate the determinants
of developer compensation on a global scale.

By collecting the voice of developers, the Stack Overflow Annual
Developer Survey 2023 enables analysts, IT leaders, reporters and other
developers to stay up to date with the latest trends and technologies
that are shaping the industry. The extensive reach and depth of
information provided by the survey not only helps and provides context
in understanding where trends in technology are heading, but also offers
key attributes that can potentially influence developer compensation.
Through a thorough analysis of this rich dataset, we can gain a deeper
understanding of the complex interplay between various variables, such
as developer roles, skills, experience, and geographic location, that
contribute to determining remuneration in the global tech industry.

\textbf{Objectives} The primary objective of this project is to conduct
a detailed exploratory data analysis (EDA) of the Stack Overflow Annual
Developer Survey 2023, focusing on understanding the factors that affect
developer compensation across the world. We aim to use compensations as
the continuous response variable in our regression models to predict
compensation based on various predictors such as geographic location,
experience, technology used, and role in the industry. Additionally, we
aim to investigate how AI/ML technologies are integrated into
developers' work processes and consider their influence on job
satisfaction and overall compensation. By analysing these, we plan to
discover insights into the career paths of developers and the evolving
technology landscape, providing valuable information for stakeholders in
the tech industry. This analysis will highlight potential fields for the
developer community and identify which variables are more influential.

\textbf{Project Rationale} The tech industry is characterized by rapid
advancements and constant change, making it imperative to understand the
determinants of developer compensation. The factors that influence
compensation have far-reaching implications, not only for individual
developers' career trajectories but also for the overall productivity
and innovation within organizations. It is therefore vital to study the
interplay between the aforementioned factors in determining
compensation.

\end{document}
